\documentclass[12pt]{article}

%%%%%%%%%%%%%%%%%%%%%%% Don't change anything in here. This space is called the preamble, it is where you tell the computer to load the proper LaTeX packages to perform the math and formatting desired. 

\usepackage{physics} 
\usepackage{siunitx} 
\usepackage{enumerate} 
\usepackage{pgfplots}
\usepackage{pgfplotstable}
\usepackage{tikz,pgfplots}
\usepackage{amsmath}  %I added this so that you can use the align tool for equations!
\usepackage{wasysym} %This package allows you to put emojis in your paper!!!!
	%wasysym: \smiley{} \frownie{} see http://milde.users.sourceforge.net/LUCR/Math/mathpackages/wasysym-symbols.pdf for list of most symbols available in this package
	
\usepackage{geometry}
 \geometry{
 a4paper,
 total={170mm,257mm},
 left=20mm,
 top=20mm,
 }

\pgfplotsset{compat=1.14}
%%%%%%%%%%%%%%%%%%%%%%%%% Again, Don't change anything Above %%%%%%%%%%%%%%%%%%%%

\begin{document}


\title{\Huge IoT Based Smart Parking System}
\author{T.RamSrikumar}
\date{\today}
\maketitle



%%%% to use this template, please copy-paste the entire thing into a new document and save it so you have it!

%%%%% If you want to omit something in this lab, place a % sign to the left of it and it won't show up on the lab, like this line!



\section{Introduction}
	 
    IoT-based smart parking solutions are intended to enhance the parking experience for locals, tourists, and business travellers. They combine IoT gadgets like cameras, ultrasonic sensors, and RFID tags to track parking place availability in real-time and direct drivers to open spots. Smart parking systems can assist drivers in finding parking spots more quickly and easily and save time and fuel by giving real-time information about parking spot availability.The growth of Internet of Things have given rise to New
    possibilities in terms of smart cities. Smart parking facilities
    and traffic management systems have always been at the core 
    of constructing smart cities.\\
    
    There are lot of benefits by implementing this Iot application:
\begin{description}
  \item[$\bullet$] No need to waste time on looking for parking.
  \item[$\bullet$] Reduction in time and fuel spent by road user 
searching for parking.
  \item[$\bullet$] Proper selection of vehicle according to the 
availability of parking space.
  \item[$\bullet$] Easy ease of payments.
\end{description}
    
    %note the cite above. That is an in text citation. By putting citations in your bibliography, you can cite in text simply using the \cite{} command and putting the number of the citation in the {} 
    

    
\section{Architecture}
	The architecture of a smart parking system is designed to collect data from parking lots and garages, process that data, and use it to improve the parking experience for citizens, visitors, and business people.\\
    \\
    The architecture of a smart parking system typically includes several layers: 

%%%%%%%%%%Here is a sample bulleted list that is often useful for the materials section. You can also number or letter your materials to easily refer to them later on
\begin{description}
  \item[$\bullet$] Sensor layer: This layer consists of a variety of sensors and actuators that are deployed in the parking lots and garages, such as ultrasonic sensors, cameras, and RFID tags, to collect data on parking spot occupancy, parking spot availability, and other parameters.
  \item[$\bullet$] Network layer : This layer connects the sensor layer to the back-end infrastructure. It is typically based on wireless technologies such as Zigbee, WiFi, and cellular networks and is responsible for transmitting data from the sensors to the back-end infrastructure.
  \item[$\bullet$] Data management layer: This layer is responsible for collecting, storing, and processing the data generated by the sensors. It typically includes databases, data analytics tools, and data visualization tools.
  \item[$\bullet$] Application layer: This layer includes the applications and services that are built on top of the data management layer. These applications and services provide city officials, residents, and businesses with access to the data and insights generated by the sensor layer.
  \item[$\bullet$] Security layer: This layer is responsible for protecting the smart parking system infrastructure from potential cyber-attacks, unauthorized access, and data breaches. It includes security technologies such as encryption, authentication, and access control.
  \item[$\bullet$] Management layer: This layer includes the management systems used to monitor and control the smart parking system infrastructure. It includes systems for fault management, configuration management, performance management, and security management.
  \item[$\bullet$] Payment layer: This layer includes the payment systems that enable the payment process, like smart parking meter, which are IoT-enabled parking meters that can accept payments through a variety of methods such as credit card, mobile payments, and can allow you to pay for parking remotely.
  \item[$\bullet$] User interface: This layer includes the interface that the end-users interact with, such as web or mobile application, interactive kiosk, or other forms of user interface that provide information about parking spot availability, parking spot location, parking prices, and allow users to make payment and check their parking status. This layer enables the end-users to access the parking system easily and efficiently, and it is an important aspect of the overall user experience.
\end{description}
%%%%%%%%%%%%%%%%%%%%%%%%%%%%%%%%%%%%%%%%%%%%%%


\section{Protocols}
In smart parking systems, various communication protocols are used to ensure the smooth and efficient functioning of the system. These protocols are used for different purposes such as device management, data communication, and security. 
\begin{description}
\item[$\bullet$] MQTT (Message Queue Telemetry Transport): A lightweight publish-subscribe protocol that is commonly used for communication between IoT devices and the back-end infrastructure.
  \item[$\bullet$] CoAP (Constrained Application Protocol): A lightweight protocol that is designed for resource-constrained devices and low-power networks, commonly used for communication between IoT devices.
  \item[$\bullet$] BACnet (Building Automation and Control Network): A communication protocol that is commonly used in building automation systems, it is designed for use in automation systems such as HVAC, lighting, and security.
  \item[$\bullet$] HTTP (Hypertext Transfer Protocol) and HTTPS (HTTP Secure) : These are the most commonly used protocols for web communication, they can be used to get parking spot availability status, price and other information over the internet.
  \\
  \\
  \\
  \\
\end{description} 



\section{My Views}
	Smart parking systems can make it easier for drivers to find available parking spots, reducing the time and fuel wasted searching for a parking spot. By dynamically adjusting parking prices based on demand, smart parking systems can encourage drivers to park in less congested areas, reducing traffic congestion in high-demand areas. Smart parking systems can also help municipalities to optimize the use of parking resources and increase revenue from parking fees. Additionally, reducing the time and fuel wasted searching for parking spots can also help to reduce the environmental impact of cars. On the other hand, some drawbacks that the public may view include privacy concerns, cost, complexity, and technical issues. The use of cameras and other sensors to monitor parking spot occupancy may raise privacy concerns among some members of the public. Implementing and maintaining a smart parking system can be costly, and municipalities may need to invest in new infrastructure and technologies. Smart parking systems can be complex, and municipalities may need to invest in training and support to ensure that they are used effectively. Smart parking systems rely on a variety of technologies and networks, and technical issues such as power outages, network failures, or software bugs can disrupt the system.  .

\end{document}